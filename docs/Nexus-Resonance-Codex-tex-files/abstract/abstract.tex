\begin{abstractbox}
	We introduce the Nexus Resonance Codex (NRC), a mathematical framework centered on exponentially weighted series involving the golden ratio $\phi = (1 + \sqrt{5})/2$, modular cycles of Fibonacci--Lucas--Pell sequences, residue-class exclusion filters modulo 9 (and extensions to 27/81), and a family of multi-scale transforms. Core contributions include:
	
	\begin{itemize}[leftmargin=*, label=\textcolor{accentorange}{\textbullet}]
		\item $\phi$-weighted Dirichlet-type series $\sum_{n=1}^\infty \chi(n) \phi^{-kn} n^{-s}$ ($k \ge 1$), which are entire functions with numerical zeros clustering near the line $\operatorname{Re}(s) \approx -\ln\phi \approx -0.48121$;
		\item A formal 3-6-9-7 modular exclusion principle that projects out terms congruent to 0, 3, 6 modulo 9, yielding series with distinct analytic behavior;
		\item Five transforms: Triple Theta Transform (TTT), Golden Tensor Theory (GTT), Multi-Scale Tensor Transform (MST), Trageser Universal Prime Transform (TUPT), and Quantum Residue Transform (QRT), each defined precisely with verifiable properties;
		\item High-dimensional extensions via sparse E8-based projections to 256D--4096D, with observed entropy scaling $H_d \approx 10.96 - \ln\phi \cdot \log d$.
	\end{itemize}
	
	All results are supported by symbolic identities, modular arithmetic verifications (periods 24/72/216), high-precision numerical computations (mpmath, $dps \ge 80$), and reproducible Python code. Empirical observations in protein core regions (12.2\% deficit in mod-9 classes, $p \approx 4.8 \times 10^{-63}$ in 34k residues) suggest modular exclusion as a potential stability filter. Applications are indicated in sparse representations, post-quantum cryptography, and lattice-based simulations. Historical heuristics are discussed in Appendix A.
	
	\vspace{1em}
	\noindent\textbf{\textsf{Keywords:}} golden ratio, Dirichlet series, modular exclusion, Fibonacci, Lucas, Pell, polylogarithm, tensor transforms, lattice projections, entropy scaling
\end{abstractbox}