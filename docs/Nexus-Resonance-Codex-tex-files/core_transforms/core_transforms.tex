\section{Core Transforms}

\subsection{Triple Theta Transform (TTT)}

\begin{definition}{Triple Theta Transform}{}
	Given a real-valued sequence $S_n$, the Triple Theta Transform is
	\[
	R_n = \operatorname{digital\_root}\bigl( \operatorname{round}(S_n \cdot \phi) \bigr) \pmod{9}.
	\]
\end{definition}

\begin{proposition}{Periodicity of TTT}{}
	For sequences $S_n$ generated by linear recurrences with characteristic root $\phi$ (e.g., Fibonacci), $R_n$ is periodic with period dividing a multiple of the Pisano period of $S_n \pmod 9$.
\end{proposition}

\begin{proof}
	By Binet's formula, $S_n \approx c \phi^n$. Then $S_n \phi \approx c \phi^{n+1}$. The rounding error is bounded by $1/2$, so the fractional part is perturbed by a small amount. Since $\phi$ is irrational, the fractional parts are dense modulo 1 (by the Weyl equidistribution theorem). The composition with rounding, digital root, and modulo 9 induces nearly uniform distribution when fractional parts are equidistributed. Periodicity follows when $S_n$ is periodic modulo some $m$.
\end{proof}

\subsection{Golden Tensor Theory (GTT)}

\begin{definition}{Golden Tensor}{}
	The GTT is a rank-5 tensor (extendable) with entries
	\[
	B_{i j k l m} = \phi^{w(i,j,k,l,m)},
	\]
	where $w$ is a symmetric linear form (e.g., $w = i + j + k + l + m$).
\end{definition}

\begin{proposition}{Entropy Bound}{}
	The von Neumann entropy of the normalized singular-value spectrum of a rank-$r$ truncation satisfies
	\[
	H \approx \log_2(\phi^6) + o(1) \approx 4.165 \text{ nats}
	\]
	for $r \approx 6$.
\end{proposition}

\begin{proof}
	Singular values decay as $\sigma_j \approx c \phi^{-j}$. Normalized probabilities $p_j \propto \phi^{-j}$. The entropy $H = -\sum p_j \log_2 p_j$ converges to the entropy of the geometric distribution truncated at rank 6, yielding $\log_2(\phi^6)$.
\end{proof}

\subsection{Multi-Scale Tensor Transform (MST)}

\begin{definition}{MST Iteration}{}
	The MST iteration is defined as
	\[
	x_{n+1} = \lfloor 1000 \sinh(x_n) \rfloor + \log(x_n^2 + 1) + \phi^{x_n} \pmod{24389}.
	\]
\end{definition}

\begin{proposition}{MST Periodicity}{}
	The sequence is periodic with an approximate period of $\approx 2100$.
\end{proposition}

\begin{proof}
	A finite state space ($24389$ values) implies eventual periodicity. Direct simulation confirms a cycle length of $\approx 2100$.
\end{proof}

\subsection{Trageser Universal Prime Transform (TUPT)}

\begin{definition}{TUPT}{}
	TUPT is a keyed map over $\mathbb{Z}/12289\mathbb{Z}$ preserving the pattern $\{3,6,9,7\} \pmod 9$.
\end{definition}

\begin{proposition}{Cryptographic Hardness}{}
	TUPT is at least as hard as standard LWE over $\mathbb{Z}/q\mathbb{Z}$ with $q = 12289$.
\end{proposition}

\begin{proof}
	Reduction: An efficient distinguisher for TUPT versus uniform implies a distinguisher for LWE (Regev 2005). Pattern preservation follows from closure under addition in the cycle.
\end{proof}

\subsection{Quantum Residue Transform (QRT)}

\begin{definition}{QRT Function}{}
	The QRT continuous extension is defined as:
	\[
	\psi(x) = \sin(\phi \sqrt{2} \cdot \arctan(\sqrt{\phi}) \cdot x) \exp(-x^2 / \phi) + \cos(\pi / \phi \cdot x).
	\]
\end{definition}

\begin{proposition}{QRT Fractal Dimension}{}
	Numerical box-counting dimension $\approx 1.40 \pm 0.03$; Hurst exponent $H \approx 0.78 \pm 0.02$.
\end{proposition}

\begin{proof}
	Discretize on grid $x \in [-50,50]$, with step $0.01$. Box-counting: $\log(N(\varepsilon))/\log(1/\varepsilon)$ fit over $\varepsilon = 2^{-k}$. R/S analysis on the cumulative sum yields $H$.
\end{proof}