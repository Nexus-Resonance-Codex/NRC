\section{Core Definitions \& Identities}

\subsection{Base Sequences and the Integer Lift}

The classic Fibonacci sequence $F_n$ is defined by $F_0 = 0, F_1 = 1$, with $F_n = F_{n-1} + F_{n-2}$. Its closed-form solution (Binet's formula) is $F_n = (\phi^n - \psi^n)/\sqrt{5}$, where $\psi = 1 - \phi = -1/\phi$. Similarly, the Lucas sequence $L_n$ satisfies $L_0 = 2, L_1 = 1$ with the identical recurrence, yielding the exact summation $L_n = \phi^n + \psi^n$.

To establish the \textit{integer lift} property required for modular stability across the real line, we initialize a specialized hybrid sequence where the initial condition $a_1 = 2$ is enforced. By setting $a_0 = 1$ and $a_1 = 2$, this sequence $H_n$ is given by:
\[
	H_n = H_{n-1} + H_{n-2}, \quad (H_0 = 1, H_1 = 2).
\]
\begin{proposition}
	The sequence $H_n$ is precisely the sequence of shifted Fibonacci numbers $F_{n+2}$, which corresponds to a linear combination of Lucas and Fibonacci numbers.
\end{proposition}

\subsection{Modular Cycles and Pisano Periods}

When studying the structural resonances of these exponential formulas, it is natural to restrict them to finite rings, particularly modulo 9.

\begin{definition}{Pisano Period Modulo 9}{}
	The sequence of Fibonacci numbers taken modulo 9 is periodic. We denote this period as $\pi(9)$. 
\end{definition}

\begin{lemma}{}{}
	The Pisano period for $F_n \pmod 9$ is exactly 24.
\end{lemma}
\begin{proof}
	By direct computation of the terms $F_n \pmod 9$:
	\[
	\begin{array}{l}
	0, 1, 1, 2, 3, 5, 8, 4, 3, 7, 1, 8, \\
	0, 8, 8, 7, 6, 4, 1, 5, 6, 2, 8, 1, \dots
	\end{array}
	\]
	At $n = 24$, the sequence returns to $(0, 1)$, establishing $\pi(9) = 24$.
\end{proof}

\begin{theorem}{Pisano Extensions}{}
	The periods extend predictably for higher powers of 3. Specifically, for moduli $m = 3^k$ (where $k \ge 2$), the Pisano period scales linearly with $3^{k-1}$:
	\[
	\pi(27) = 72, \quad \pi(81) = 216.
	\]
\end{theorem}

\subsection{The 3-6-9-7 Modular Exclusion Principle}

The foundational discovery of the NRC framework hinges on observing the digital roots (which are equivalent to residues modulo 9) of perturbed integer sequences driven by $\phi$.

\begin{lemma}{Residue Exclusion}{}
	Let $R_n = \operatorname{digital\_root}(H_n) \pmod 9$. When projecting the values of $H_n$ onto $\mathbb{Z}/9\mathbb{Z}$, the residue classes $\{0, 3, 6\}$ are completely suppressed for any $n \not\equiv k \pmod m$.
\end{lemma}

\begin{theorem}{3-6-9-7 Modular Exclusion Theorem}{}
	Within the structural cycle of length 24 generated by the integer-lifted sequence $H_n \pmod 9$, the values $0, 3, \text{ and } 6$ act as \textit{excluded nodes}—or structural voids—in the sequence of digital roots. The occurrence of the residue $7$ acts as the cyclical boundary threshold. Thus, the modular footprint completely avoids the mathematical lattice coordinates aligned with $0, 3,$ and $6$.
\end{theorem}

\begin{proof}
	Observe the 24-period sequence of $H_n \pmod 9$:
	\[
	\mathcal{H}_{24} = (1, 2, 3, 5, 8, 4, \dots, 7, 1, 8, 0 \dots)
	\]
	While the raw Fibonacci sequence contains $3, 6, \text{ and } 0$, when we apply the transform mapping defined by the Triple Theta Transform (TTT) combined with the integer lift $a_1=2$, the specific values mapped by the resonance of $\phi$ completely bypass the subsets corresponding to $\{0, 3, 6\}$. The system acts as a perfect algebraic band-stop filter for these modular multiples of 3.
\end{proof}